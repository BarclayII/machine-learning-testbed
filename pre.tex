\documentclass[10pt,CJK]{beamer}
\usetheme{Singapore}

\usepackage[noindent,UTF8]{ctexcap}
\usepackage[utf8]{inputenc}
\usepackage[T1]{fontenc}
\usepackage{amsmath}
\usepackage{amsfonts}
\usepackage{amssymb}
\usepackage{graphicx}
\usepackage{textcomp}

\author{甘全}
\title{强化学习与对象追踪及检测}
%\subtitle{}
%\logo{}
%\institute{}
%\date{}
%\subject{}
%\setbeamercovered{transparent}
%\setbeamertemplate{navigation symbols}{}

\begin{document}
	\maketitle
	
	\begin{frame}
		\frametitle{内容概要}
		\begin{itemize}
			\item 问题的提出
			\item 强化学习简介
			\begin{itemize}
				\item 马尔可夫决策过程
				\item Q-Learning与Sarsa
				\item “强化”算法系列(REINFORCE Algorithms)
			\end{itemize}
			\item 所谓“深度强化学习”
			\begin{itemize}
				\item 深度Q-Learning(DeepMind, 2013)
				\item 循环注意力模型(Recurrent Attention Model)
				(DeepMind, 2014)
			\end{itemize}
			\item Tracking as a game
			\item 新手上路,多多包涵
		\end{itemize}
	\end{frame}
	
	\begin{frame}
		\frametitle{问题的提出---~Learning Where \& What
			\only<8->{+ What to Do}}
		\pause
		\begin{itemize}
			\item 从视频中对物体进行追踪\pause
			\begin{itemize}
				\item 静止、匀速直线、圆周、抛物线、随机游走……\pause
				\item 从各个角度、各种距离去观察\pause
				\item 成功以任何角度、任何距离追踪某个物体的任何运动\pause
				\item 就算提取到该物体的特征(?)\pause
			\end{itemize}
			\item 将在视频中提取得到的特征用在静态图像上,
			以进行对象检测(?)\pause
			\item 对被追踪/检测的对象的特征(或行为等)作决策\pause
			\begin{itemize}
				\item 分类/标注\pause
				\item 视角转动\pause
				\item 镜头平移、伸缩
			\end{itemize}
		\end{itemize}
	\end{frame}
	
	\begin{frame}
		\frametitle{一个尝试}
		\begin{itemize}
			\item 如果把检测、追踪、分类、决策的过程当作一个单人游戏……
			\pause
			\begin{itemize}
				\item 设计合适的运行规则和奖励惩罚
				\pause
				\item 让计算机去玩这个游戏
				\pause
				\item 让计算机找到解这个游戏的最佳策略
				\pause
				\item 问题就能得到解决
			\end{itemize}
			\pause
			\item 强化学习(Reinforcement Learning)
			\pause
			\begin{itemize}
				\item 如何设计游戏规则\pause
				\item 如何设计计算机寻找该游戏最佳策略的算法
			\end{itemize}
		\end{itemize}
	\end{frame}
	
	\begin{frame}
		\frametitle<1-5>{什么是游戏}
		\begin{itemize}
			\item \only<2-6>{\alert{状态}(state),可以分两类:
			\begin{enumerate}
				\item 内部状态(内部数据或任何不可为玩家所知的信息)
				\item 可见状态(视觉、听觉、感觉等)
			\end{enumerate}}
			\only<7->{状态集合:
			\begin{enumerate}
				\item 内部状态集合$S$
				\item<alert@11|uncover@11> 可见状态集合$\Omega$
			\end{enumerate}}
			\item \only<3-7>{可供选择的\alert{决策或行动}(action)}
			\only<8->{可供选择的决策或行动集合$A$}
			\item \only<4-8>{状态间的\alert{转移规则}
			\begin{itemize}
				\item 大部分情况下,转移不是唯一的
				\item 可见状态也有相应的转移规则
			\end{itemize}}
			\only<9->{状态转移函数
				$P_a\left(s\right)\colon S\times A\mapsto S$
			\begin{itemize}
				\item 更准确地说,是一个分布列或者分布密度函数
				$P_a\left(s,s'\right)$
				\item<alert@11|uncover@11> 可见状态转移函数
				$\Pi_a\left(\omega,\omega'\right)$
			\end{itemize}}
			\item \only<5-9>{状态转移时得到的\alert{收益}(reward)规则}
			\only<10->{收益函数
				$R_a\left(s,s'\right)\colon S\times A\times S\mapsto \mathbf{R}$}
		\end{itemize}
		\frametitle<6->{马尔可夫决策过程(Markov Decision Process)}
		\frametitle<11->{\alert{部分可见}
			马尔可夫决策过程(Partially Observable MDP)}
	\end{frame}
	
	\begin{frame}
		\frametitle{求解MDP——动态规划法}
		\begin{itemize}
			\item 求解MDP,即寻找一个最优\alert<1>{策略}(policy)
			$\pi\left(s\right)\colon S\mapsto A$
			\begin{itemize}
				\item 能够在所有状态下找到使收益期望最大的行动
			\end{itemize}\pause
			\item 最优策略的“子策略”也是最优的(最优子结构)\pause
			\item 动态规划
			\begin{itemize}
				\item 策略函数,当前状态下期望收益最高的行动\\
				$\pi\left(s\right)=
				\arg\max_a\lbrace
				\sum_{s'}
				\alert<6>{P}_a\left(s,s'\right)\left[
				\alert<6>{R}_a\left(s,s'\right)+
				\only<5->{\alert<5>{\gamma}}V(s')
				\right]
				\rbrace$
				\item 价值函数,当前状态下按最优策略可得到的期望总收益\\
				\only<1-3>{
				$V\left(s\right)=
				\sum_{s'}P_{\pi\left(s\right)}
				\left(s,s'\right)\left[
				R_{\pi\left(s\right)}\left(s,s'\right)+
				\only<5->{\gamma}V\left(s'\right)
				\right]$}
				\only<4->{
				$V\left(s\right)=\max_{a}
				\sum_{s'}\alert<6>{P}_a\left(s,s'\right)\left[
				\alert<6>{R}_a\left(s,s'\right)+
				\only<5->{\alert<5>{\gamma}}
				V\left(s'\right)\right]
				$}
			\end{itemize}\pause\pause
			\item \alert<5>{衰减因子(discount factor)
				$\gamma\in\left[0,1\right)$}
			\begin{itemize}
				\item 展开后有$\gamma^n$项,在步数足够大时趋近于0,可收敛
			\end{itemize}\pause
			\item 新问题:
			\begin{itemize}
				\item \alert<6>{$P$与$R$一般未知}
				\item 需要已知状态集合$S$及终止状态,计算顺序从后往前
			\end{itemize}
		\end{itemize}
	\end{frame}
		
	\begin{frame}
		\frametitle{求解MDP——蒙特卡罗法}
		\begin{itemize}
			\item 随机初始化一个策略$\pi$\pause
			\item 进行多次游戏,获得各个状态下采用各个行动时的期望总收益
			\begin{itemize}
				\item 此即\alert<2>{质量函数}\\
				$Q\left(s,a\right)=
				\sum_{s'}P_{a}\left(s,s'\right)\left[
				R_{a}(s,s')+\gamma V\only<3->{\alert<3>{_\pi}}(s')
				\right]$
			\end{itemize}\pause\pause
			\item 将每一个$\pi\left(s\right)$更新为
			$\arg\max_{a}Q\left(s,a\right)$\pause
			\item 如是重复,直至收敛\pause
			\item 不再依赖$P$和$R$\pause
			\item 计算顺序从前到后,不需要已知终止状态\pause
			\item 问题:
			\begin{itemize}
				\item 效率太低
				\item 不适用于长期甚至不终止游戏
				\item 状态集合有限小(尽管不必全部已知)
			\end{itemize}
		\end{itemize}
	\end{frame}
	
	\begin{frame}
		\frametitle{Q-\only<1-9>{Learning}\only<10>{Network}}
		\begin{itemize}
			\item $\alert<10>{Q
			\only<3->{\alert<3>{_{i+1}}}
			\left(s,a\right)}=
			\only<1-6>{\sum_{s'}P_{a}\left(s,s'\right)}
			\only<7->{\alert<7>{\alpha}}
			\left[
			\alert<10>{
			\only<1-5>{R_{a}(s,s')}
			\only<6->{\alert<6>{r}}
			+\gamma
			\only<1>{V(s')}
			\only<2->{\alert<2>{\max_{a'}Q
			\only<3->{\alert<3>{_{i}}}
			\left(s',a'\right)}}}
			\right]
			\only<7->{\alert<7>{+\left(1-\alpha\right)
			Q_i\left(s,a\right)}}
			$ \pause\pause\pause
			\item 随机选取一个初始状态$s_0\in S$\pause
			\item 对当前状态$s$,(按某一分布)选取一个合适的行动$a$
			\begin{itemize}
				\item 一般按$\epsilon$-贪心($\epsilon$-greedy)分布,
				Softmax等
			\end{itemize}\pause
			\item 观察下一状态$s'$和收益$r$\pause
			\item 结合新的$Q$值和原有的$Q$值,更新$Q\left(s,a\right)$
			\begin{itemize}
				\item 引入学习率$\alpha$,达到平均效果
			\end{itemize}\pause
			\item 如是重复,直至收敛\pause
			\item 应用:对每一当前状态$s$,
			选取令$Q\left(s,a\right)$最大的$a$,不再随机
			\begin{itemize}
				\item 属于“策略分布不相关”(off-policy)学习算法
			\end{itemize}\pause
			\item 可以用神经网络
			$Q^*\left(s,a;\theta\right)$
			去逼近$Q$,以适应状态和行动规模较大的问题
			\begin{itemize}
				\item 训练:最小化
				$r+\gamma\max_{a'}Q^*\left(s',a';\theta_i\right)$
				与
				$Q^*\left(s,a;\theta_{i+1}\right)$的误差平方
			\end{itemize}
		\end{itemize}
	\end{frame}
	
	\begin{frame}
		\frametitle{SARSA\only<10>{-Network}}
		\begin{itemize}
			\item $
			Q_{i+1}\left(s,a\right)=
			\alpha\left[
			r+\gamma\only<1>{\alert<1>{\max_{a'}}}
			Q_i\left(s',a'\right)
			\right]+
			\left(1-\alpha\right)Q_i\left(s',a'\right)
			$\pause\pause
			\item 随机选取一个初始状态$s_0\in S$\pause
			\item\alert<4>{按某一分布选取合适的初始行动$a_0\in A$}
			\begin{itemize}
				\item 一般按$\epsilon$-贪心($\epsilon$-greedy)分布,
				Softmax等
			\end{itemize}\pause
			\item 观察下一状态$s'$和收益$r$\pause
			\item\alert<6>{在状态$s'$下按同样分布选取合适的行动$a'$}\pause
			\item 更新$Q$,\alert<7>{并令$s\gets s',a\gets a'$}\pause
			\item 如是重复,直至收敛\pause
			\item 应用:对每一当前状态$s$,
			\alert<9>{按同样分布选取合适的行动$a$},
			而非选取使$Q$最大的$a$
			\begin{itemize}
				\item 属于“策略分布相关”(on-policy)学习算法
			\end{itemize}\pause
			\item 也可以用神经网络去逼近$Q$,以解决规模较大的问题
		\end{itemize}
	\end{frame}
	
	\begin{frame}
		\frametitle{Deep Q-Learning: Playing Atari with RL}
		\begin{itemize}
			\item 游戏进程特别长(至少数千帧,甚至不会结束)
			\begin{itemize}
				\item 样本前后依赖关系较强
				\item 在线学习时,距离当前时间较近的历史样本作用较大
			\end{itemize}\pause
			\item 在Q-Network的基础上加入\alert{经验池}(experience pool)
			\pause
			\item 每转移一次状态,算法将转移事实$(s_t,a_t,r_t,s_{t+1})$
			加入经验池中
			\begin{itemize}
				\item 状态$s_t$为最近数帧图像的叠加
			\end{itemize}\pause
			\item 算法从经验池中随机抽样调整权重
			\begin{itemize}
				\item 有效利用距离当前相对久远的样本,打破依赖关系
				\item 允许使用SGD调整权重
			\end{itemize}\pause
			\item “记忆”并非存在于网络的某一组件中(如RNN、LSTM),
			而是被整合在网络权重中\pause
			\item 算法采用CNN作为函数逼近\pause
			\item 拿“深度学习”的架构去做“强化学习”的事
		\end{itemize}
	\end{frame}
	
	\begin{frame}
		\frametitle{Deep Q-Learning的开源实现}
		\begin{itemize}
			\item Kristjan Korjus: Replicating DeepMind
			\item Andrej Karpathy: ConvNetJS,在一个简化场景下\pause
			\begin{itemize}
				\item 训练一个agent,在一个有障碍物(墙壁)的地图中寻找苹果、
				并回避毒蘑菇
				\item 输入并非图像的叠加,而是在某个视角内有9个传感器,
				以此作为原始的输入并进行叠加
				\item 决策有前进,小左转,大左转,小右转,大右转五种
				\item 前进距离固定(适当小),可能的朝向只有24个
			\end{itemize}
		\end{itemize}
	\end{frame}
	
	\begin{frame}
		\frametitle{决策方式众多的情况}
		\begin{itemize}
			\item 以上各个方法的局限:决策数不能太多\pause
			\item 决策数如果很多(或者决策空间连续),构造网络就比较困难\pause
			\begin{itemize}
				\item 选定一个位置\pause
				\begin{itemize}
					\item 在一个稍大的网格中选取位置就已经相当困难
					\item DeepMind(Jun. 2014)提到多于25个离散位置的选择
					策略学习起来较为困难
				\end{itemize}\pause
				\item 确定发射角度、力度等\pause
			\end{itemize}
		\end{itemize}
	\end{frame}
	
	\begin{frame}
		\frametitle{REINFORCE}
		\begin{itemize}
			\item Williams(1992)概括了一类称为REINFORCE的算法
			\item 适用于游戏环境内部状态转移与当前采取行动无关的情况
			\pause
			\begin{itemize}
				\item 例子:物体跟踪
				\item 此时,只需考虑当前时刻的收益
				(立即收益,immediate reward)
				\item 单纯地累加所有时刻的最大立即收益即得到最大总收益\pause
				\item 状态转移与玩家采取行动有关的情况下,
				只考虑立即收益的贪心思想应该也是近似地求最优解的一个方法
				\begin{itemize}
					\item MNIST,包括被污染的情况(Cluttered MNIST)
					(DeepMind, Jun. 2014)
					\item 接球游戏(DeepMind, Jun. 2014)
					\item 门牌号识别(DeepMind, Dec. 2014)
				\end{itemize}
			\end{itemize}
		\end{itemize}
	\end{frame}
	
	\begin{frame}
		\frametitle{REINFORCE}
		\begin{itemize}
			\item$
			\only<1>{\text{REward Increment}}
			\only<2->{\Delta w_{ij}}
			=
			\only<1-2>{\text{Non-negative Factor}}
			\only<3->{\alpha_{ij}}
			\times
			\only<1-3>{\text{Offset Reinforcement}}
			\only<4->{\left(r-b_{ij}\right)}
			\times
			\only<1-4>{\text{Characteristic Eligibility}}
			\only<5->{\dfrac{\partial\ln g_i}{\partial w_{ij}}}
			$\pause
			\item $w$:权重\pause
			\item $\alpha$:学习率\pause
			\item $r$:当前收益
			\item $b$:基线,理论上可任取,但收敛效果差异很大,
			有时可能影响最终结果\pause
			\item $g$:不同决策取值的概率密度函数\pause
			\item Williams证明了:上式得出的权重变化的期望
			$\mathbf{E}\left(\Delta w\vert w\right)$,
			与收益期望关于权重的梯度
			$\nabla_w\mathbf{E}\left(r\vert w\right)$的内积非负
			\begin{itemize}
				\item 特别地,若$\alpha$恒为一常数,则
				$\mathbf{E}\left(\Delta w\vert w\right)=
				\alpha\nabla_w\mathbf{E}\left(r\vert w\right)$
			\end{itemize}
		\end{itemize}
	\end{frame}
	
	\begin{frame}
		\frametitle{循环注意力模型(Recurrent Attention Model)}
		\begin{itemize}
			\item 解决MNIST分类和接球问题\pause
			\item 可见状态:视野传感器(glimpse sensor)以选取点为中心,
			由内而外提取各个层次各个清晰度的图像
			\only<2>{
			\begin{itemize}
				\item 视野(glimpse)大小由内而外倍增,越外层越模糊
				\item 模拟人类的视觉
			\end{itemize}}\pause
			\item 决策:选择下一个视野中心,以及其它与问题相关的具体决策
			\only<3>{
			\begin{itemize}
				\item MNIST分类:决定所属的数字
				\item 接球:移动接球板,有左移、右移和不动三种
			\end{itemize}}\pause
			\item 收益:0-1标记,成功为1,失败为0
			\only<4>{
			\begin{itemize}
				\item MNIST分类:分类正确为1,分类失败为0
				\item 接球:成功接住为1,漏球为0
				\item 收益非常稀疏,不到最后不出结果
			\end{itemize}}\pause
			\item 内部状态转移:受决策影响有限
			\only<5>{
			\begin{itemize}
				\item MNIST:内部状态为整张静止图片,恒等转移
				\item 接球游戏:内部状态为球和板的位置,由于收益稀疏,可以考虑
				用立即收益的叠加来近似求解
			\end{itemize}}\pause
			\item 问题转化成POMDP
		\end{itemize}
	\end{frame}
	
	\begin{frame}
		\frametitle{RAM的架构}
		\begin{figure}
			\centering
			\includegraphics[scale=0.3]{network.png}
		\end{figure}
		\begin{itemize}
			\only<1>{
			\item 各层次视野与当前选取中心点一同作为输入,输入到一个RNN中
			\begin{itemize}
				\item 在真正被RNN接收前,
				各视野的组合首先被Linear-rectifier变换成一个高维向量,
				随后与由中心位置经变换得到的高维向量连接,作为整体再经过
				一层全连接的Linear-rectifier处理,得到最终传递给RNN的输入
				\item MNIST分类采用了RNN,接球游戏中采用了LSTM
			\end{itemize}}\pause
			\only<2>{
			\item RNN处理输入得到输出}\pause
			\only<3>{
			\item 输出被分别传递到位置输出网络和决策输出网络中
			\begin{itemize}
				\item 位置网络为一普通的线性变换
				\item 决策网络是一个单层Linear-softmax网络
			\end{itemize}}\pause
			\only<4>{
			\item 以输出的位置为均值,给定方差,作一二元正态分布,
			并按此分布取位置作为下一时刻的视野中心
			\begin{itemize}
				\item 方差太小会导致agent无法有效探索更远的位置,从而
				不能利用在外层角落里的信息;方差太大会降低学习效率
			\end{itemize}}
		\end{itemize}
	\end{frame}
	
	\begin{frame}
		\frametitle{RAM的训练}
		\begin{itemize}
			\item 目标:最大化收益期望\pause
			\item 采用REINFORCE算法
			\begin{itemize}
				\item $
				\Delta\theta=\alpha\sum_{t=1}^T\lbrace
				\left(R_t-b_t\right)
				\left[\nabla_{\theta}\ln g\left(a_t,l_t\vert
				s_{1:t};\theta
				\right)\right]\rbrace$
				\item $R_t$为至$t$时刻的总收益
				\item $b_t$为基线,仅和RNN的内部状态有关,与实际的行动
				$a_t,l_t$无关
			\end{itemize}\pause
			\item 针对问题
			\begin{itemize}
				\item 分类:对每一张图提取固定个数的视野,随后分类,得到收益
				\item 接球:每一时刻提取一个视野,直至成功接球或漏球
				\begin{itemize}
					\item 游戏时间比分类要长得多,故采用记忆时间较长的LSTM
				\end{itemize}
				\item 门牌号识别:采用能检测多个对象的Deep RAM
			\end{itemize}
		\end{itemize}
	\end{frame}
	
	\begin{frame}
		\frametitle{Tracking as a game}
		\begin{itemize}
			\item 用RAM模型单纯做对球(单个像素点)的追踪\pause
			\item 游戏环境:$24\times 24$纯黑背景,
			单个白点作为要追踪的对象\pause
			\item 球的运动:自上而下,到最下一行时结束
			\begin{itemize}
				\item 匀速直线运动,碰到垂直墙壁反弹
				\item 一定限制下的变速运动,除碰到垂直墙壁反弹之外:
				\begin{itemize}
					\item 水平方向上的速度每一时刻变化
					-0.5\textasciitilde0.5 pixels per frame,
					\item 垂直方向上速度每一时刻变化
					0\textasciitilde0.2 pixels per frame
				\end{itemize}
				\item 静止不动,此时球的出现位置随机。仅用在测试中
			\end{itemize}\pause
			\item Agent训练:
			$6\times 6,12\times 12,24\times 24$三层视野
			\begin{itemize}
				\item RAM + REINFORCE,如果最内层视野中有球,收益为1,
				否则为0
				\item RAM + REINFORCE + 经验池,将历史游戏记录
				保存在经验池中然后做SGD,对学习效果影响不大
				\item LSTM,网络结构同RAM,但不用REINFORCE训练,而是
				直接以视野中心与球的平均距离作为误差函数进行权重调整
			\end{itemize}\pause
			\item 交叉测试:
			把在某一环境中训练得到的网络应用在另外的环境中
		\end{itemize}
	\end{frame}
	
	\begin{frame}
		\frametitle{Tracking as a game}
		\begin{figure}
			\centering
			\begin{tabular}{|c|c|c|c|}
				\hline & 匀速 & 变速 & 静止 \\ 
				\hline LSTM-匀速 & 92.22\% & 83.14\% & 50.3\% \\ 
				\hline LSTM-变速 & 86.73\% & 84.87\% & 46.8\% \\ 
				\hline RAM-匀速 & 72.10\% & 44.66\% & 71.0\% \\ 
				\hline RAM-变速 & 41.68\% & 61.04\% & 56.8\% \\ 
				\hline 
			\end{tabular}
		\end{figure}
		\begin{itemize}
			\item LSTM + Distance训练样本个数为5000
			\item RAM + REINFORCE训练样本个数60000,方差0.07
			\item 测试样本数1000
			\item 匀速和变速环境中的指标为成功追踪率=成功追踪帧数\textdiv 总帧数
			\item 静止环境中的指标为检测率=至少一帧追踪成功的样本数\textdiv 总样本数。
			样本数量1000,测试时长50帧。
		\end{itemize}
	\end{frame}
	
	\begin{frame}
		\frametitle{Tracking as a game}
		\begin{itemize}
			\item 总体而言,RAM + REINFORCE在单纯追踪中的学习效率和成功率不如
			LSTM + Distance。
			\begin{itemize}
				\item RAM + REINFORCE在静态环境中交叉测试表现意外地比
				LSTM + Distance效果要好一些
			\end{itemize}\pause
			\item 强化学习可能的优势:Tracking, action一步到位?普适性更强?
			\pause
			\item 进一步的计划:
			\begin{itemize}
				\item 更多环境设置(从不同方向出现,同时训练不同轨迹,
				加入障碍物等)\pause
				\item 加入背景(引入CNN?)\pause
				\item 更改被追踪的对象形式(刚体,周期变化的对象(?)等)\pause
				\item 让agent自行模拟匀速运动的球
				\begin{itemize}
					\item 是否能记住墙的位置
				\end{itemize}
				\item 多个对象的追踪(DRAM)
				\begin{itemize}
					\item 如何对照
				\end{itemize}\pause
				\item 再现DeepMind的接球游戏
				\begin{itemize}
					\item 如何对照
				\end{itemize}
			\end{itemize}
		\end{itemize}
	\end{frame}
	
	\begin{frame}
		\frametitle{Discussion}
	\end{frame}
\end{document}